\documentclass{llncs}
%
\usepackage{makeidx}
\usepackage{hyperref}
\usepackage{listings}
\usepackage{scrextend}
\addtokomafont{labelinglabel}{\sffamily}
\usepackage{xcolor}
\usepackage[spanish]{babel}
\selectlanguage{spanish}
\usepackage[utf8]{inputenc}
%
\begin{document}
%
\frontmatter
%
\pagestyle{headings}
\addtocmark{Informe TP Funcional}

\title{Informe TP Funcional}
%
\titlerunning{Informe TP Funcional}
%
\author{De León, Lautaro Nahuel}
%
\institute{Universidad Nacional de La Plata}
%
\maketitle

\begin{abstract}
  Este trabajo muestra una implementación sencilla del juego de ajedrez, con una interfaz simple y compacta, reglas intuitivas y fácilmente extensibles y una inteligencia artificial sencilla y elegante.
\end{abstract}
%
\section{Introducción}
%

La idea del presente trabajo es recrear un juego de ajedrez que permita realizar enfrentamientos contra oponentes humanos y openentes simulados mediante IA. El espíritu de este desarrollo es mantener en todo momento, y en la medida de lo posible, la simpleza tanto en el diseño como en la implementación de la solución, de manera tal de poder explicar con relativa fluidez los conceptos involucrados -especialmente en la IA- y que sirva de modelo inicial para futuras iteraciones más sofisticadas.

\subsection{Motivación}
\label{subsec:Motivacion}

La motivación es, principalmente, la suma de dos inquietudes: por un lado, el arrancar un proyecto medianamente serio usando conceptos puros de programación funcional -y Haskell como medio para cumplir estrictamente con dicha premisa-; y por otro lado, el entender de manera clara cómo son los algoritmos de IA y tratar de aislar esta lógica del resto de la aplicación de manera tal de poder generalizarla y usarla potencialmente en otros juegos.

Posteriormente, la idea es publicar el proyecto en un repositorio abierto (por ejemplo, Github) para que la comunidad pueda aprovecharlo, e idealmente, extenderlo con nuevas ideas, correcciones o mejoras.

\subsection{Estructura del trabajo}
\label{subsec:Estructura trabajo}

A lo largo del trabajo se tratará de mostrar una una línea de afuera hacia adentro, es decir, desde el funcionamiento de la interfaz con el usuario, pasando por su implementación, hasta llegar a los confines de las reglas del juego y del desarrollo de la IA.

Dicho eso, la primera sección explicará el funcionamiento detallado de la interfaz, desde el punto de vista de un usuario.

En la segunda sección se mostrará el diseño y la implementación del sistema; primero por la estructura de todos los tipos definidos y, posteriormente, por el módula de la interfaz con el usuario, avanzando hacia los módulos del juego y del tablero. Luego se desarrollará la teoría y la implementación del motor de IA en detalle, y al final de la sección, una optimización por paralelización.

La tercera sección planteará el desarrollo de algunos tests basados en propiedades.

Y finalmente, en la cuarta sección se expondrán las conclusiones.

\section{Presentación e interfaz de la aplicación}

Ante todo, es relevante mencionar que se colocó más énfasis en la lógica del juego y en la modularización del mismo de forma tal que el desarrollar otra GUI más atractiva sea una tarea relativamente sencilla. En otro apartado se detallará la estructura interna de la aplicación.

Ahora sí, respecto a la presente interfaz se puede decir que es textual y la configuración depende de los parámetros pasados al programa, el cual funciona como un comando UNIX. A continuación se presentará una salida típica de la ayuda que brinda esta utilidad:

\begin{lstlisting}[frame=single, language=bash, caption=Ayuda de la aplicación, label={lst:ayuda_programa}]
> ./chesssimple --help
The Chesssimple program

Chesssimple [COMMAND] ... [OPTIONS]
  Chess game & engine

Common flags:
  -? --help            Display help message
  -V --version         Print version information

Chesssimple hvh [OPTIONS]

     --p1name=ITEM     The player 1's name
     --p1color=ITEM    The player 1's color
     --p2name=ITEM     The player 2's name

Chesssimple [hvc] [OPTIONS]

     --hname=ITEM      The human player's name
     --hcolor=ITEM     The human player's color
  -c --cstrength=INT   The computer player's strength

Chesssimple cvc [OPTIONS]

     --p1strength=INT  The player 1's strength
     --p2strength=INT  The player 2's strength
\end{lstlisting}

La aplicación dispone de 3 modos de juego: \textit{humano vs computadora} -el cual es el modo por defecto-, \textit{humano vs humano} y \textit{computadora vs computadora}. Cada uno de los modos cuenta además con parámetros de configuración específicos para elegir con qué bando jugar -blancas o negras- y, para el caso de los jugadores por CPU, la dificultad o fuerza con la que contará. Vale la pena mencionar que todos estas opciones cuentan también con valores predefinidos.

Una vez iniciada la aplicación, se verá la interfaz principal de la siguiente manera:

\begin{lstlisting}[frame=single, language=bash, caption=Interfaz principal del programa, label={lst:interfaz_principal}]
Welcome to Simple Chess Game

( -T  _ -B -Q -K -B -N -T )
( -p -p -p -p -p -p -p -p )
(  _  _ -N  _  _  _  _  _ )
(  _  _  _  _  _  _  _  _ )
(  _  _  _ +p  _  _  _  _ )
(  _  _  _  _  _  _  _  _ )
( +p +p +p  _ +p +p +p +p )
( +T +N +B +Q +K +B +N +T )

White moves...
Commands are: exit, which, move, undo
_
\end{lstlisting}

Lo que se representa es el tablero y las piezas a través de una matriz de caracteres, donde el significado para cada una de ellas es:
\newline

\setlength{\tabcolsep}{20pt}
\begin{tabular}{ l | c }
  Caracter & Significado \\
  \hline
  T & Torre \\
  N & Caballo \\
  B & Alfil \\
  Q & Dama \\
  K & Rey \\
  p & Peón \\
\end{tabular}
\newline
\newline

Y los colores se representan como:
\newline

\setlength{\tabcolsep}{20pt}
\begin{tabular}{ l | c }
  Caracter & Significado \\
  \hline
  + & Blanco \\
  - & Negro \\
\end{tabular}
\newline
\newline

En cuanto a la notación posicional se puede mencionar que es por coordenadas; es decir, cada casillero o escaque se identifica por el número de fila -entre 1 y 8, creciente hacia arriba- y por la letra de columna -entre \textit{a}  y \textit{h}, creciente hacia la derecha-. A modo de ejemplo, en la posicional inicial de ajedrez (dada la perspectiva del tablero en \ref{lst:interfaz_principal}), el peón blanco movió desde d2 a d4 y el caballo negro movió desde b8 a c6.

La mecánica del juego consiste, para un jugador humano, en introducir comandos. Los disponibles actualmente son:
\newline

\begin{labeling}{which}
  \item [move] Este es claramente el comando principal. Permite indicar una posición de partida y una de destino utilizando la notación previamente comentada. Por ejemplo, \textit{move d2 d4} mueve el peón blanco de la columna \textit{d}  dos escaques hacia adelante.

    Si la posición es inválida (porque está mal escrita, fuera de los límites del tablero, o el casillero no tiene una ficha correspondiente al bando propio), el comando se cancela.
  \item[which] Dada la posición de una pieza del bando propio, lista todas las posiciones a las cuales podría moverse. Similar al \textit{move}, no dará ningún resultado si la posición es inválida. Además tampoco dará resultados si la posición está en jaque y la pieza candidata a mover no puede contrarrestar esa situación.

    Por ejemplo, para el alfin blanco ubicado en c1 en la figura \ref{lst:interfaz_principal}, los movimientos disponibles arrojados por el comando son: d2, e3, f4, g5, h6
  \item[undo] Deshace un turno completo, es decir, el turno del rival y el propio. Por ejemplo, para la situación planteada en la figura \ref{lst:interfaz_principal}, ejecutar \textit{undo} llevaría el juego a la posición inicial nuevamente.
  \item[exit] Sale del programa.
\end{labeling}

Si ambos jugadores son humanos, cada uno dispondrá de esta interfaz para introducir sus movimientos. En cambio, si juega la computadora, se mostrará un cartel de espera mientras la CPU procesa su jugada, hasta realizar su movimiento.

Para cualquiera de los modos, el juego se desarrolla normalmente. Los casos especiales a destacar son:

\begin{enumerate}
  \item \textbf{Jaque}: Tal situación es informada por el programa en el momento en que un bando debe realizar su jugada y, en concordancia con las reglas del ajedrez, sólo se permitirá el movimiento de aquellas piezas que hagan que dicho bando salga del jaque. En otras palabras, cualquier otra jugada que no pueda cancelar el jaque, es considerada inválida.
  \item \textbf{Jaque mate}: Continuando la definición anterior, si se detecta que no hay jugadas para salir del jaque, automáticamente el juego termina con jaque mate y un cartel informando que el bando que tenía que jugar perdió la partida.
\end{enumerate}

\section{Diseño e implementación}

En esta sección se dará un pantallazo general y completo de la arquitectura de la aplicación. A grandes rasgos, el proyecto se divide lógicamente en:

\begin{itemize}
  \item Módulo principal de interfaz.
  \item Módulos de lógica y reglas del juego.
  \item Módulo de IA
  \item Tests
\end{itemize}

Cada una de las partes será explicada detalladamente en las próximas secciones. Vale recalcar que esta división se justifica en el hecho de que cada una de estas partes pueden ser sustituidas por otras sin afectar al resto del sistema. Como ejemplos, se pueden mencionar:

\begin{itemize}
  \item Cambiar el módulo de interfaz por una API web que sirva el estado de la aplicación y sea mostrado por un cliente corriendo en un navegador web.
  \item Cambiar, sumar o corregir reglas y/o estrategias del juego sin que impacte demasiado en el resto de los componentes.
  \item Cambiar y/o mejorar el motor y los algoritmos de IA con el objeto de tener distintos o mejores resultados para los contrincantes por CPU de forma totalmente transparente al funcionamiento del juego.
  \item Sumar o corregir tests en pos de mejorar la cobertura o la calidad de los mismos.
\end{itemize}

\subsection{Estructura del sistema según los tipos}

Antes de explicar cada una de las partes conformantes de la aplicación, será conveniente mostrar los tipos algebraicos empleados con el fin de mostrar un panorama general del sistema -y especialmente de la lógica del juego- para luego facilitar el desarrollo puntual de cada uno de los módulos.

Como se anticipó en la introducción del presente informe, la idea fue pensar y modelar una representación sencilla del juego.

El módulo del juego es, valga la redundancia, \textit{Game}. La forma de su tipo es:

\begin{lstlisting}[frame=single, language=haskell, captionpos=b, caption=Tipo de Game, label={lst:tipo_game}]
data Game = Game { player1 :: Player.Player
                 , player2 :: Player.Player
                 , plays   :: [Board.Board]
                 , turn    :: Color.Color
                 }
\end{lstlisting}

Tal como se observa, es un registro (o \textit{product type}) que encapsula una partida de ajedrez: hay dos jugadores -\textit{player1} y \textit{player2}-, una lista de tableros (\textit{Board}) para representar la traza de las jugadas y un indicador de turno (a través de \textit{Color}) para saber qué jugador es el próximo en mover.

Los jugadores (\textit{Player}) son de la siguiente forma:

\begin{lstlisting}[frame=single, language=haskell, captionpos=b, caption=Tipo de Player, label={lst:tipo_player}]
data Player =
  HumanPlayer    { name     :: String,  color :: Color } |
  ComputerPlayer { strength :: Integer, color :: Color }
  deriving (Show)
\end{lstlisting}

El jugador es una unión (o \textit{sum type}) de registros en donde la idea es diferenciar a los jugadores humanos y a los jugadores por computadora, no sólo por los atributos de cada uno, sino además, por la lógica particular de cada uno al momento de interactuar con el juego.

El último tipo relevante es el tablero (\textit{Board}) que, en realidad, va a contener otros tipos internos, a saber:

\begin{lstlisting}[frame=single, language=haskell, captionpos=b, caption=Tipo de Board y derivados, label={lst:tipo_board}]
type Board         = Matrix Square
data Square        = BlankSquare |
                       OccupiedSquare ColouredPiece
                         deriving (Eq)
data ColouredPiece = CP (Color, Piece) deriving (Eq)
data Piece         = Pawn  | Knight | Bishop | Tower |
                     Queen | King deriving (Eq)
type Position      = (Int, Int)
\end{lstlisting}

El tablero es una matriz de casilleros o escaques (\textit{Square}), en donde cada uno de ellos, o bien, está vacío (\textit{BlankSquare}), o bien, está ocupado por una pieza de color, esta última representada por una tupla cuyo primer componente es el color al bando al que pertenece (\textit{Color}) y cuyo segundo componente es la pieza en sí (\textit{Piece}). La posición es simplemente una coordenada de números para ubicar casilleros en el tablero.

\subsection{Módulo principal (\textit{Main.hs})}

A continuación se arrancará con el módulo de inicio de la aplicación. Como ya se ha mencionado, este componente se encarga de presentar una interfaz de interacción hacia el usuario, y para ello, se hace uso exhaustivo de la entrada/salida de la consola; consecuentamente -desde Haskell- se utiliza la mónada IO.

El punto de entrada \textit{main} es el siguiente:

\begin{lstlisting}[frame=single, language=haskell, captionpos=b, caption=Punto de entrada de la aplicación, label={lst:punto_entrada}]
main :: IO ()
main = do
  arguments <- cmdArgs (modes [humanvshuman,
    humanvscomputer &= auto, computervscomputer] &=
    program "Chesssimple" &= help "Chess game & engine")
  Screen.reset
  Screen.printWithColor "Welcome to Simple Chess Game"
    "white"
  let (player1, player2) = playersFromArgs arguments
      game               = Game.new player1 player2
   in performGame game
\end{lstlisting}

Todo \textit{main} en Haskell es de tipo IO () dado que es un programa ejecutable y, por tanto, recibe y envía información desde y hacia el mundo exterior, en particular, el sistema operativo que lo invoca.

Está escrita con \textit{do-notation} dado que se realizan varios pasos que imprimen alguna clase de efecto lateral sobre el denominado mundo exterior: la primera línea hace uso de una función \textit{cmdArgs} de un paquete homónimo cuyo próposito es procesar los argumentos que recibe el programa. En este caso particular, recibe en ellos las características de los jugadores que participarán en la partida, las cuales se han descrito previamente (ver figura \ref{lst:ayuda_programa}). Posteriormente se usan esos datos para crear un juego con dos jugadores e iniciar la partida.

También se pueden ver en el medio algunas funciones accesorias como \textit{Screen.reset} y \textit{Screen.printWithColor}, que limpian la consola e imprimen letreros formateados con colores, respectivamente. Dichas funciones se encuentran encapsuladas en el módulo \textit{Screen}, y todas ellas operan con la salida a través de la mónada IO.

El flujo de ejecución sigue con \textit{performGame}:

\begin{lstlisting}[frame=single, language=haskell, captionpos=b, caption=Función performGame, label={lst:perform_game}]
performGame :: Game.Game -> IO ()
performGame game = do
  printGameLayout game
  case Game.isCheckMate game of
    True  -> Screen.printWithColor
      ("Game finished! " ++ colorTurn game ++ " loses!")
        "red"
    False -> performGameTurn game $ Game.whoPlaysNow game
\end{lstlisting}

Nuevamente se trabaja con la IO para imprimir el tablero en \textit{printGameLayout}. Esta función aprovecha que el tipo \textit{Game} y todos los tipos que los componenn son instancias de la clase \textit{Show} y, por ende, tienen una representación textual.

Luego comprueba si el juego está en situación de «jaque mate», en cuyo caso, termina la partida, y si no, pasa a evaluar la función \textit{performGameTurn}:

\begin{lstlisting}[frame=single, language=haskell, captionpos=b, caption=Función performGameTurn, label={lst:perform_game_turn}]
performGameTurn :: Game.Game -> Player.Player -> IO ()
performGameTurn game (Player.ComputerPlayer strength _) =
  let updatedGame = GameAI.performMovement game strength
   in performGame updatedGame
performGameTurn game (Player.HumanPlayer name _)    = do
  putStrLn "Commands are: exit, which, move, undo"
  userInput <- getLine
  case parseCommand userInput of
    ("exit" ,         _) -> return ()
    ("which",     pos:_) -> do
      Screen.printWithColor ("Available movements are: " ++
        (showAvailableMovements game (parsePosition pos)))
          "white"
      Screen.pause
      performGame game
    ("move" , src:dst:_) -> do
      performMoveAction game (parsePosition src)
        (parsePosition dst)
    ("undo" ,         _) -> performUndoAction game
    _ -> do
      Screen.printError "Bad command. Try again."
      performGame game
\end{lstlisting}

Esta función hace uso del \textit{pattern matching} para discernir qué tipo de lógica ejecutar de acuerdo al tipo particular de un jugador: si es una computadora, delegará la jugada al motor de IA (\textit{GameAI}) -el cual se explicará en una sección posterior-, y si es un humano, hará uso una vez más de la entrada/salida para obtener un comando de la consola, procesarlo y determinar la acción a ejecutar en función de él. Todo esto es lo que se puede ver en la estructura \textit{case} de la función. Para cada uno de ellos se recuperan, opcionalmente, datos de utilidad y se suministran a las acciones correspondientes. Dado que el \textit{parseo} del comando puede fallar, se utiliza la mónada \textit{Maybe} para modelar tal efecto.

Para concluir con este apartado, se mostrará a continuación la acción correspondiente al comando más relevante, \textit{move}:

\begin{lstlisting}[frame=single, language=haskell, captionpos=b, caption=Función performMoveAction asociada al comando move, label={lst:perform_move_action}]
performMoveAction :: Game.Game -> Maybe Position ->
                       Maybe Position -> IO ()
performMoveAction game Nothing dst = do
  Screen.printError "Bad source position"
  performGame game
performMoveAction game src Nothing = do
  Screen.printError "Bad destiny position"
  performGame game
performMoveAction game (Just src) (Just dst) =
  case (Game.tryMovement game src dst) of
    Just nextGame -> performGame nextGame
    Nothing -> do
      Screen.printError "Illegal movement. Try again."
      performGame game
\end{lstlisting}

Esta acción sólo se comunicará con el módulo principal del juego -\textit{Game}- en caso de tener validadas sintácticamente las posiciones de origen y de destino para mover una pieza. En cualquier otro caso se informará la condición del error.

En caso de éxito se repetirá la ejecución de \textit{performGameTurn} con el estado del juego actualizado. En los casos de error se volverá a pedir un nuevo comando al usuario con exactamente el mismo juego. En cualquiera de los casos se puede ver que hay conformado un ciclo de acciones que solamente finaliza cuando se detecta que el juego ha terminado.


\subsection{Módulos de lógica y reglas del juego}

En esta sección se comenzará a explicar los aspectos más relevantes de la lógica del juego.

Siguiendo con la perspectiva \textit{top-down}, se arrancará por el módulo \textit{Game}. Las funciones que se exportan son las siguientes: \textit{new, isInitial, isCheck, isCheckMate, tryMovement, currentBoard, whoPlaysNow, update, turn, show, availableMovements, undo, player1, player2}. Muchas de estas funciones son observadoras, es decir, permiten consultar el estado actual del juego (¿quién juega en este turno?, ¿hay jaque?, ¿hay jaque mate?, ¿qué movimientos posibles existen en determinada posición?), y otras, en cambio, son funciones que alteran de alguna manera el estado presente del juego actual y devuelven una nueva instancia del juego, tal como el caso de \textit{update} o de \textit{tryMovement}. Esta última es una de las más relevantes porque es la que acciona un turno en una partida y es la primera entrada a la lógica del juego, tal como se puede dilucidar del código expuesto en \ref{lst:perform_move_action}. A continuación, se mostrará el cuerpo de dicha función:


\begin{lstlisting}[frame=single, language=haskell, captionpos=b, caption=Función tryMovement, label={lst:try_movement}]
tryMovement :: Game -> Board.Position ->
               Board.Position -> Maybe Game
tryMovement game src dst =
  case Board.movePiece
    (currentBoard game) (turn game) src dst of
      Just newBoard -> Just (update game newBoard)
      Nothing       -> Nothing
\end{lstlisting}

Dada una partida, una posición de origen y una posición de destino, la función intentará realizar el movimiento delegando todas las validaciones pertinentes a la función \textit{Board.movePiece} con la información actual del juego, es decir, el tablero y el turno actual. El resultado final es de tipo \textit{Maybe Game} dado que el movimiento podría no ser válido.

Probablemente el módulo más importante del juego sea \textit{Board}: en él se encuentra codificada la mayor parte del funcionamiento del juego. Una vez más, se comenzará la exploración por la última función que se ha visto, \textit{Board.movePiece}


\begin{lstlisting}[frame=single, language=haskell, captionpos=b, caption=Función movePiece, label={lst:move_piece}]
movePiece :: Board -> Color -> Position ->
             Position -> Maybe Board
movePiece board turn src dst =
  let availableMovements = freeMovements board turn src
   in if dst `elem` availableMovements
      then Just $
        treatPromotions $ placePiece board src dst
      else Nothing
\end{lstlisting}

Esta función recibe un tablero, un color que representaría el turno del bando al cual le corresponde jugar, una posición de origen y una posición de destino. Su manera de trabajar se basa en calcular los movimientos libres que se disponen desde la posición origen del tablero y si la posición destino se encuentra entre tales movimientos se procederá a realizar el movimiento en \textit{placePiece} (esta función simplemente reubicará la pieza escogida en la posición de destino en la matriz). También se puede ver la función \textit{treatPromotions} la cual detecta si un peón llegó a la fila del enemigo, en cuyo caso, se transformará en una dama; si no se da tal caso, se dejará el tablero intacto.

Cuando la posición de destino no se encuentra entre los movimientos disponibles desde la posición de origen, el tablero no se actualizará con el movimiento pretendido y la función evaluará a Nothing, es decir, no se corresponderá a un tablero válido para el movimiento pretendido. Las razones de por qué el movimiento es inválido se comenzará a fundamentar en la definición de la siguiente función, \textit{freeMovements}:


\begin{lstlisting}[frame=single, language=haskell, captionpos=b, caption=Función freeMovements, label={lst:free_movements}]
-- in this function there is control
-- if the kings position is in check or not.
freeMovements :: Board -> Color -> Position -> [Position]
freeMovements board turn position =
  filter (\possibleDst -> not $
   isCheck (placePiece board position possibleDst) turn) $
      uncheckedFreeMovements board turn position
\end{lstlisting}

Esta función, en realidad, delega gran parte de la lógica de validación del movimiento a otra función llamada \textit{uncheckedFreeMovements}. La razón de esto es que la presente función necesita saber si los movimientos posibles a través de la posición de origen provocan un jaque sobre el rey propio. De ser así, el movimiento no podría realizarse y, como consecuencia, debe descartarse.

Tras conocer dicha restricción, ahora se mostrará la función \textit{uncheckedFreeMovements}:


\begin{lstlisting}[frame=single, language=haskell, captionpos=b, caption=Función uncheckedFreeMovements, label={lst:unchecked_free_movements}]
-- 'unchecked' means that there is no control
-- if the kings position is in check or not.
uncheckedFreeMovements :: Board -> Color ->
                          Position -> [Position]
uncheckedFreeMovements board turn position
  | not (legalGrab board turn position) = []
  | isKnightOccupied board position     =
      knightFreeMovements  board turn position
  | isPawnOccupied board position       =
      pawnFreeMovements    board turn position
  | otherwise                           =
      generalFreeMovements board turn position
\end{lstlisting}

De aquí lo que se controla, en primer lugar, es si la posición de origen es <<legal>>, es decir, si tal posición del tablero está ocupada por una pieza del bando que tiene que mover en dicho turno; en contraposición, el inicio del movimiento será ilegal si la posición está fuera del tablero, si corresponde a un casillero desocupado o si corresponde a un casillero ocupado por una pieza enemiga.

Luego, se considera que el caballo y el peón tienen movimientos especiales por sobre el resto de las piezas (el caballo es la única pieza que puede <<saltear>> piezas, y el peón captura de una forma distinta a la cual se mueve); por tal razón, se tratan sus movimientos en funciones diferenciadas. En cualquier otro caso, se emplea la función \textit{generalFreeMovements}:


\begin{lstlisting}[frame=single, language=haskell, captionpos=b, caption=Función generalFreeMovements, label={lst:general_free_movements}]
generalFreeMovements :: Board -> Color ->
                        Position -> [Position]
generalFreeMovements board turn position =
  concatMap (availabilityPositionFilter board turn)
    (allMovements colouredPiece position)
  where
    selectedSquare = square board position
    colouredPiece  = piece selectedSquare
\end{lstlisting}

Su funcionamiento se divide en dos partes: por un lado se tiene la generación del movimiento a partir del tipo de pieza seleccionada (\textit{allMovements}); y por otro lado se tiene la puesta de la pieza en el contexto del tablero, a través de la función \textit{availabilityPositionFilter}. La presentación de esta última deberá demorarse unos párrafos puesto que será necesario explicar otras funciones previamente. En cambio, se comenzará a desarrollar la última de las funciones, es decir, \textit{allMovements}:

\begin{lstlisting}[frame=single, language=haskell, captionpos=b, caption=Función allMovements, label={lst:all_movements}]
allMovements :: ColouredPiece -> Position -> [[Position]]
allMovements colouredPiece position =
  map (filter isInsideBoard) $
    pieceUnboundedMovements colouredPiece position
\end{lstlisting}

Lo que simplemente hace es recortar el movimiento de las piezas según los límites del tablero. En un primer momento se pensó en hacer que \textit{pieceUnboundedMovements} fuera una función infinita de manera tal que la dimensión del tablero no dependiera del movimiento de las piezas, pero dicho objetivo se abandonó rápidamente en favor de cumplir con el alcance del trabajo. De cualquier manera la implementación está abierta a realizar dicho cambio o mejora, y ninguna otra parte del código debería verse afectada, excepto por \textit{isInsideBoard}.

Antes de volver con el desarrollo de \textit{generalFreeMovements}, se mostrarán las funciones \textit{pieceUnboundedMovements} y \textit{towerMovements} con el propósito de ir cerrando la exploración del módulo:

\begin{lstlisting}[frame=single, language=haskell, captionpos=b, caption=Función pieceUnboundedMovements, label={lst:piece_unbounded_movements}]
pieceUnboundedMovements :: ColouredPiece -> Position ->
                           [[Position]]
pieceUnboundedMovements (CP (Black, Pawn)) =
  blackPawnMovements
pieceUnboundedMovements (CP (White, Pawn)) =
  whitePawnMovements
pieceUnboundedMovements (CP (_, Tower))    =
  towerMovements
pieceUnboundedMovements (CP (_, Knight))   =
  knightMovements
pieceUnboundedMovements (CP (_, Bishop))   =
  bishopMovements
pieceUnboundedMovements (CP (_, Queen))    =
  queenMovements
pieceUnboundedMovements (CP (_, King))     =
  kingMovements
\end{lstlisting}

\begin{lstlisting}[frame=single, language=haskell, captionpos=b, caption=Función towerMovements, label={lst:tower_movements}]
towerMovements :: Position -> [[Position]]
towerMovements (i,j) =
  let up    = [ (x, y) | x <- reverse [1..i-1], y <- [j] ]
      down  = [ (x, y) | x <- [i+1..8], y <- [j] ]
      left  = [ (x, y) | x <- [i], y <- reverse [1..j-1] ]
      right = [ (x, y) | x <- [i], y <- [j+1..8] ]
   in [ up, down, left, right ]
\end{lstlisting}

Un detalle importante a notar es que todas estas funciones tienen tienen tipo \textbf{[[Position]]}. Esto es porque hay una división implícita entre las direcciones de movimientos de una pieza -para el caso de la torre sería arriba, abajo, izquierda, derecha- y los movimientos en sí que puede realizar en cada una de esas direcciones. Este segundo nivel tiene que ser simpre monotónico, y es importante recalcarlo porque es una precondición necesaria para la primera parte de la función \textit{generalFreeMovements}, representada por la función \textit{availabilityPositionFilter} que, ahora sí, podrá desarrollarse:


\begin{lstlisting}[frame=single, language=haskell, captionpos=b, caption=Función availabilityPositionFilter, label={lst:availability_position_filter}]
availabilityPositionFilter :: Board -> Color ->
                              [Position] -> [Position]
availabilityPositionFilter board turn positions =
  let (frees, occupied) =
         span (isSquareFree board) positions
   in if (not . null) occupied &&
        isPositionOccupiedByEnemy board turn
          (head occupied)
      then (head occupied):frees
      else frees
\end{lstlisting}

El funcionamiento de este <<filtro>> es el siguiente: dada una lista de posiciones, la cual representa un vector de movimiento de una pieza -con la precondición de que sea monotónicamente creciente o decreciente sobre la extensión del tablero, como ya se ha mencionado- lo que se hace es calcular una bi-partición cuya frontera estará dada por la ocupación de un casillero por una pieza (esto se realiza con la función \textit{span}).  Tras esto se tendrán los casilleros desocupados a los cuales la pieza de partida podrá ubicarse. Debería quedar claro en este punto la importancia de la propiedad monotónica de la lista; de lo contrario, no podría garantizarse que esas posiciones libres están realmente antes de la primera posición detectada como <<ocupado>> (en tal sentido, \textit{span} es una función similar a \textit{takeWhile}).
Adicionalemente, si el primer casillero ocupado se encuentra habitado por una pieza enemiga, dicha posición se agregará también al listado de posiciones disponibles puesto que es capturable.

Con esto se considera concluida la explicación de, al menos, la columna vertebral de la lógica del juego que corresponde principalmente al movimiento de las piezas en el tablero, respetando las reglas básicas del ajedrez.


\subsection{Módulos de IA}

Una parte importante de este trabajo se encuentra materializada en los módulos de IA que permiten que el CPU <<juegue>> con las reglas del ajedrez. Antes de desarrollar los detalles de la implementación, se necesitarán algunos conceptos teóricos sobre los algoritmos de IA utilizados.

\subsubsection{ Teoría / Background }

El ajedrez, como juego, presenta dos características importantes:

\begin{enumerate}
  \item Es un juego de \textit{información perfecta}.
  \item Es un juego de \textit{suma cero}.
\end{enumerate}

Un juego de información perfecta es aquel en el cual toda la información necesaria para tomar una decisión -mover una pieza, por ejemplo- se encuentra de alguna manera visible en el entorno del juego -para el caso del ajedrez, el entorno sería el tablero-. La importancia de destacar esto es que no existen elementos libres al azar a la hora de evaluar el contexto del juego, y como consecuencia directa, la IA no tendrá lógica que dependa de la aleatoriedad.

Un juego de suma cero es aquel en donde el impacto de una decisión tomada por un jugador afectará de forma inversamente proporcional al otro o al resto de los jugadores, con lo cual, la suma del valor cuantitativo del efecto particular de cada decisión sobre cada jugador es siempre cero. Por ejemplo, en el caso del ajedrez, si un jugador A realiza un movimiento que lo deja en un contexto favorable, necesariamente dejará en un contexto desfavorable, y con el mismo valor de intensidad, a un jugador B. En contrarrecíproca, no pueden existir jugadas o decisiones que sean favorables o desfavorables para ambos jugadores al mismo tiempo.

Esta última característica es fundamental para esbozar el algoritmo que es el núcleo de la IA desarrollada en este trabajo, que es el algoritmo \textit{minimax}

\subsubsection{ Algoritmo Minimax }

Para describir este algoritmo, primero hace falta construir un \textbf{árbol de jugadas} (en la bibliografía también aparece como \textbf{árbol de búsqueda}). La idea es que a partir de un juego -o de su entorno actual, como por ejemplo, el presente tablero- se puedan calcular todas las posibles jugadas -cada una de ellas representadas por un nodo- que un jugador podría ejecutar y, para cada una de ellas, calcular todas las jugadas posibles que el adversario podría realizar en contestación. Este procedimiento debería terminar cuando se llega una situación terminal, en la cual, el juego termina.

Adicionalemente se requiere una \textbf{función de evaluación}. La idea de esta función heurística es otorgarle un puntaje al nodo presente -o jugada calculada- de manera tal que un número positivo implique que la posición es favorable para un jugador A y, en cambio, un número negativo implique que la posición es desfavorable para el mismo jugador. Lógicamente, considerando que se trata de un juego de suma cero, si un número positivo es favorable para A, entonces será proporcionalmente desfavorable para un jugador B y, con un número negativo, será viceversa.

La mecánica básica de este algoritmo es que si ambos jugadores realizan sus mejores jugadas -bajo los criterios de la función de evaluación- la evaluación minimax sobre una determinada jugada será aquella que represente la mínima pérdida sobre las mejores jugadas del contrincante siguiendo exáctamente el mismo razonamiento (de ahí el nombre \textit{minimax}).

Idealmente, la función de evaluación se aplicaría sobre un nodo terminal, con lo cual, lo que importarían son tres valores: +1 para la victoria, 0 para el empate y -1 para la derrota. Dado cualquier nivel \textit{n} en el juego -nuevamente, en condiciones ideales no importaría cuál- un valor minimax positivo implicaría que tras una serie de movimientos perfectos tanto por el jugador a mover/decidir como el contrincante conducirían irremediablemente a la victoria del primero, por los supuestos que impone minimax.

Por supuesto que en la práctica esto -al menos para el ajedrez- es imposible ya que se estima que la cantidad de jugadas posibles en dicho juego es de $10^{40}$ (10 representaría el factor de ramificación, es decir, la cantidad promedio de jugadas posibles y 40 la profundidad o la duración de la partida en turnos) y en la actualidad no hay forma de computar semejante cantidad de partidas. Por lo tanto, lo que se suele realizar es recortar el árbol de búsqueda en un determinado nivel \textit{n} -y este valor \textit{n} será denominado la profundidad de búsqueda del contrincante de CPU y, más coloquialmente, su dificultad- y emplear la función de evaluación en los nodos terminales del último nivel, la cual deberá ser bastante más compleja porque en muy pocos casos se tratarán de nodos en donde el juego se considere terminado.


\subsubsection{ Poda alfa-beta }

Hay jugadas que rápidamente pueden ser descartadas. Intuitivamente, es el caso de jugadas que conducen a posiciones muy malas tal que seguir desarrollando las consecuencias de las mismas redunda en una perdida de tiempo porque ya queda en evidencia que la continuación de la jugada seguirá por otra rama totalmente distinta. Esta es la idea que se pretende capturar con la poda alfa-beta: es una técnica de optimización que, como tal, consigue el mismo resultado que el algoritmo minimax estándar pero con muchos menos cálculos.

En $\alpha$ se mantiene el mejor valor para el jugador que se pretende maximizar y en $\beta$ se mantiene el mejor valor para el jugador que se pretende minimizar. Con lo cual, si uno de los nodos que se están calculando escapan de las cotas establecidas por $\alpha$ o $\beta$, la búsqueda podría ser abandonada porque ya existen valores mejores que los que se están intantando hallar. Esto es lo que se denominda <<poda>>

Para aclarar un poco el panorama, a continuación se mostrará un ejemplo:

\begin{figure}
\begin{verbatim}
                                    [max]
                                   /     \
                                  /       \
                                [min]     [min]
                               /  \        / \
                              /    \      /   \
                             1      2    0     ?
\end{verbatim}
\caption{Ejemplo de árbol minimax con poda $\alpha\beta$}
\label{fig:ascii-box}
\end{figure}

En este ejemplo, se puede ver cómo el valor minimax del subárbol izquierdo es 1. Dado que la búsqueda es de tipo DFS, en ese momento exacto el valor de la raíz \textit{max}, así como su valor $\alpha$ es 1. Esto quiere decir que el mejor valor hasta ahora para el jugador que maximiza es 1. Siguiendo con el mismo criterio de búsqueda, se evalua que el primer hijo del subarbol derecho tiene un valor minimax 0 (el valor $\beta$ de dicho subarbol será también 0 en ese momento). Lo destacable ahora es que el subarbol \textit{min} tendrá un valor minimax $x \leq 0$. Por lo tanto, ya no tiene sentido seguir explorando el resto de los hijos de dicho subarbol porque el nodo \textit{max} seguro tendrá un valor $y \geq 1$, y eso se detecta comparando los valores actuales contra la cota de $\alpha$ -para el caso de los minimizadores- o contra la cota de $\beta$ -para el caso de los maximizadores-.

\subsubsection{ Implementación }

Para el presente trabajo se intentó abstraer y encapsular los conceptos básicos sobre un juego de suma cero -en particular, esta implementación del ajedrez- separando totalmente esas ideas de las reglas del juego. En efecto, se podría exportar lo desarrollado sobre IA aquí a otro juego de tipo suma cero, y tendría la misma validez que tiene aquí.

Para llevar a cabo dicha separación, se decidió explorar un poco lo que ofrece Haskell sobre las clases. Se modeló la clase \textit{ZeroSumGame} que cuenta con la siguiente firma:

\begin{lstlisting}[frame=single, language=haskell, captionpos=b, caption=Definición de la clase ZeroSumGame, label={lst:zero_sum_game_def}]
class ZeroSumGame zsg where
  nextGames     :: zsg -> [zsg]
  evaluateGame  :: zsg -> Int
\end{lstlisting}

Lo que se pretendió capturar con esta definición fue la teoría expuesta anteriormente: por un lado, la generación de todas las posibles jugadas para construir el árbol de búsqueda (sólo la generación, el árbol es parte de la lógica de IA) y, por otro lado, la función de evaluación para el juego. Estas dos funciones son las que se dejan como <<punto de enganche>> para que cualquier implementación de cualquier juego de suma cero las defina y obtenga, a cambio, una función \textit{performMovement} que intentará realizar el mejor movimiento posible, según la teoria ya vista. Esta función \textit{performMovement} es la que realiza el movimiento para jugadores de CPU, y puede verse presente en el código de \ref{lst:perform_game_turn}

Antes de pasar a ver los detalles de la implementación de \textit{performMovement}, a modo de documentación se deja la definición de \textit{Game} como instancia de \textit{ZeroSumGame}:


\begin{lstlisting}[frame=single, language=haskell, captionpos=b, caption=Definición de Game como instancia de ZeroSumGame, label={lst:zero_sum_game_instance}]
instance GameAI.ZeroSumGame Game where
  evaluateGame game =
    BoardAI.evaluateBoard (currentBoard game) (turn game)
  nextGames game =
    let allPossibilities =
          Board.possibleMovementsForeachTeamPosition
            (currentBoard game) (turn game)
     in catMaybes $
          concatMap (\(src, dsts) ->
            map (\dst -> tryMovement game src dst) dsts)
              allPossibilities
\end{lstlisting}

Como se puede ver, en el módulo \textit{BoardAI} se encuentra la función que realiza la evaluación del estado actual del juego, y del él, obtiene un puntaje. Los detalles internos de dicha función son propios del juego y se pueden estudiar, analizar y/o importar desde fuentes externas\footnote{Una de dichas fuentes, por ejemplo, es \url{https://chessprogramming.wikispaces.com/Simplified+evaluation+function}}. \textit{nextGames} calcula todos los juegos inmediatos posibles que se pueden obtener a partir del juego actual. \textit{catMaybes} es necesaria para <<extraer>> o ignorar los valores contenidos en la mónada Maybe de la función \textit{tryMovement}.

Ahora sí, la definición de \textit{performMovement} es:

\begin{lstlisting}[frame=single, language=haskell, captionpos=b, caption=Definición de performMovement, label={lst:perform_movement}]
performMovement :: (ZeroSumGame zsg) =>
                   zsg -> Integer -> zsg
performMovement game strength =
  bestGame $
    [(possibleGame, computeNegamax strength possibleGame) |
      possibleGame <- nextGames game]
  where bestGame =
    fst.minimumBy
      (\(_,score1) (_,score2) -> compare score1 score2)
\end{lstlisting}

Hacer el movimiento es obtener el mejor juego posible de las siguientes jugadas inmediatas \textit{nextGames} donde el mejor juego será el que posea el menor valor \textit{negamax} asociado. \textit{Negamax} es una implementación más concisa de \textit{minimax} que funciona con una función de evaluación relativa. Esto significa que la positividad o la negatividad del valor entero dependerá del jugador a quien le toque mover en el correspondiente juego, es decir, un valor positivo dirá siempre que la posición será favorable a quien le corresponda mover y un valor negativo, lo contrario. Dado que los siguientes juegos se evaluan en función del adversario -porque el turno le correspondería a él-, el menor de esos valores será el que más lo afectará y, por tanto, el mejor movimiento para el bando que está justamente realizando dicho cálculo.

\textit{computeNegamax} es de la siguiente forma:

\begin{lstlisting}[frame=single, language=haskell, captionpos=b, caption=Definición de computeNegamax, label={lst:compute_negamax}]
computeNegamax :: (ZeroSumGame zsg) =>
                  Integer -> zsg -> Int
computeNegamax depth game =
  negamax $ fmap evaluateGame $ (choptree depth) $
    gametree game
\end{lstlisting}

Dado un juego, computa su árbol de jugadas, lo recorta a la profundidad deseada (como ya se ha dicho, esto puede verse como la dificultad del CPU), luego aplica un \textit{map} de \textit{evaluateGame} sobre dicho árbol y finalmente lo reduce a su valor negamax correspondiente. Lo notable de esta implementación es que \textit{gametree game} podría ser un árbol inmensamente extenso (y hasta incluso, infinito). Y aun con el recorte propuesto con \textit{choptree depth} el árbol podría ser, de cualquier manera, muy grande. Aquí, gracias al \textit{lazy-evaluation} que posee Haskell, ninguna de las expresiones dispuestas se calcula hasta que realmente sean necesarias de computarse, y eso recién sucederá en la función \textit{negamax}:

\begin{lstlisting}[frame=single, language=haskell, captionpos=b, caption=Definiciones de negamax]
negamax :: (Tree Int) -> Int
negamax = maximum.negamax'

negamax' :: (Tree Int) -> [Int]
negamax' (Node x []) = [x]
negamax' (Node _ lt) =
  fmap negate $ mapmax (fmap negamax' lt)
\end{lstlisting}

Claramente se ve cómo en los nodos hoja se computa recién toda la cadena de computaciones vistas en el párrafo previo y  en los nodos intermedios se ignora completamente (notar el \_)

En los nodos hoja se realiza la evaluación heurística del escenario actual del juego. En los intermedios, siguiendo la estructura del árbol, se aplica recursivamente la función sobre las ramas hijas. \textit{(fmap negamax' lt)} es de tipo [[Int]], y dado un nivel $k$, representa -de forma \textit{lazy}-  el cálculo de \textit{negamax} para el nivel $k+1$. Dado que el cálculo es en DFS (primero en profundidad, de izquierda a derecha) existe cierta probabilidad creciente de que a medida que se obtengan los valores negamax de la izquierda, algunas de las ramas de la derecha ya no requieran computarse y, por tanto, puedan descartarse, según el criterio de la poda $\alpha\beta$. \textit{mapmax} intentá expresar dicha intención de la siguiente manera:

\begin{lstlisting}[frame=single, language=haskell, captionpos=b, caption=Definición de mapmax]
mapmax :: [[Int]] -> [Int]
mapmax []       = []
mapmax (xs:xss) = let max = maximum xs
                   in max:(prune max xss)
\end{lstlisting}

De cada subárbol del nival $k+1$ se va obteniendo el mejor valor y se pasa dicho valor como cota de poda para los siguientes subárboles a calcular. La poda será:

\begin{lstlisting}[frame=single, language=haskell, captionpos=b, caption=Definición de prune]
prune :: Int -> [[Int]] -> [Int]
prune bound [] = []
prune bound (xs:xss)
  | isOutsideBound xs bound = prune bound xss
  | otherwise               = mapmax (xs:xss)

isOutsideBound :: [Int] -> Int -> Bool
isOutsideBound [] _         = False
isOutsideBound (x:xs) bound =
  if bound <= x then True
                else isOutsideBound xs bound
\end{lstlisting}

En \textit{prune} se puede ver que si dentro de xs -que representaría los valores negamax de un subárbol en el nivel $k+1$- algún valor se encuentra fuera de la cota, entonces la exploración de dicho subárbol puede abandonarse puesto que ningún otro valor en él podrá mejorar lo previamente hallado en otro subárbol, determinado por dicha cota.

Sin más, esto es la IA del juego.

\subsubsection{ Paralelización }

Como detalle adicional -para mejorar los tiempos de ejecución de la IA- se ha estudiado implementar una sencilla paralelización en el algoritmo. Se ha observado que la función \textit{performMovement} del código \ref{lst:perform_movement}, computa una lista con el valor negamax de cada jugada inmediata. Como dichos cálculos son independientes entre sí, se ha contemplado realizar esa computación en paralelo con el propósito de obtener un mejor \textit{speedup}\footnote{\textit{Speedup} se define como el cociente entre un tiempo de ejecución secuencial y un tiempo de ejecución paralelo. Con $n$ hilos de ejecución, el valor del \textit{speedup}, en condiciones ideales, sería $n$, demostrando que la variante paralela aprovecha eficientemente para el cálculo a todos los hilos de ejecución.} usando varios hilos de ejecución.

La nueva versión de \textit{performMovement} se ve así:

\begin{lstlisting}[frame=single, language=haskell, captionpos=b, caption=Definición paralela de performMovement]
performMovement :: (ZeroSumGame zsg) =>
                   zsg -> Integer -> zsg
performMovement game strength =
  bestGame (allPossibleGames `using` parList rdeepseq)
  where bestGame         =
          fst.minimumBy
            (\(_,score1) (_,score2) ->
              compare score1 score2)
        allPossibleGames =
          [(aGame, computeNegamax strength aGame) |
            aGame <- nextGames game]
\end{lstlisting}

En realidad, el único cambio significativo respecto a la vieja versión es el uso del paralelismo en la expresión \textit{(allPossibleGames `using` parList rdeepseq)}. Esta <<sintaxis>> proviene del paquete \textit{parallel}\footnote{\url{https://hackage.haskell.org/package/parallel-3.2.1.1/docs/Control-Parallel-Strategies.html}} de Haskell y hace uso de estrategias de paralelización para separar la lógica real del programa del mecanismo o modelo de ejecución paralela.

La paralelización en Haskell requiere entender algunos conceptos teóricos nuevos; no por el lenguaje en realidad, sino por la naturaleza de la evaluación \textit{lazy}. Básicamente, y sin entrar en grandes detalles, existen los conceptos de \textit{normal form} y de  \textit{weak head normal form}. El primero se aplica a aquellas expresiones que son completamente evaluadas o, dicho de otra manera, que son reducidas hasta la expresión más atómica posible. El segundo se aplica sobre expresiones en dónde únicamente se evalua hasta un <<paso>> o lo que dicte el contructor más externo. En general, Haskell funciona de esta última manera, lo cual es muy útil -prueba de esto es lo anteriormente visto de la función \textit{computeNegamax} en \ref{lst:compute_negamax}-. Pero hay casos como en la paralelización en donde, por ejemplo, los elementos de una lista, como en este caso, deben ser completamente evaluados si se desea obtener una mejora importante en los tiempos de ejecución; de los contrario, lo que se paraleliza es solamente el desarrollo del <<primer nivel>> de una expresión en \textit{weak head normal form}, y pierde todo el sentido la optimización.

Para forzar la evaluación completa de la expresión se usa la estrategia \textit{rdeepseq} que internamente usa la función \textit{seq} para realizar esta acción sobre todos los componentes internos del dato. Esto, no lo puede hacer automáticamente con todos los tipos. Para ello, se requiere que estos sean instancias de la clase \textit{NFData} (afortunadamente, sí se puede derivar automáticamente sobre los tipos deseados). Con esto se indica que un determinado tipo puede evaluarse completamente hasta su reducción más atómica, y que dicho procedimiento es finito. Lógicamente, aquellos tipos que detonen alguna clase de estructura infinita, por ejemplo, no podrán ser miembros de esta clase porque, de lo contrario, jamás podrán ser evaluados completamente.

Para concluir con este experimento, se dejará una pequeña métrica empírica. Se ejecutó la solución secuencial y paralela probando con uno, dos y cuatro hilos, con una profundidad de búsqueda de cuatro. Obviamente no representa una prueba muy rigurosa, pero puede arrojar una primera pista acerca de la efectividad de la parelalización:
\\\

\setlength{\tabcolsep}{20pt}
\begin{tabular}{ l | c | r }
  Cantidad de hilos & Tiempo & Speedup \\
  \hline
  N = 1 (solución secuencial) & 51s & 1\\
  N = 2 (solución paralela)   & 26s & 1.96\\
  N = 4 (solución paralela)   & 17s & 3
\end{tabular}
\newline



Se puede notar que el speedup tiene un decaimiento logarítmico a medida que se incrementa el número de hilos, lo cual es consistente debido a los costos extras de mantener muchos de ellos en el sistema.


\section{Tests}

Los tests se realizaron con el framework QuickCheck, el cual permite definir y ejecutar tests basados en propiedades: esto es, en vez de escribir el test con los datos de entrada definidos por el usuario, se describen, por un lado, generadores para esos tipos de datos y, por otro lado, propiedades a validar. Luego, el propio framework probará con muchos datos aleatorios las propiedades descritas para determinar si se cumplen o no, donde los datos siguen la estructura definida por los generadores. Idealmente una propiedad puede considerarse verdadera si se valida para todos los elementos posibles de un dominio, pero dado que pueden ser infinitos, las pruebas automatizadas simplemente testean con muchos elementos y así puede establecerse una tendencia acerca de la <<validez>> de la propiedad.

Para el presente trabajo se comenzó la descripción de unas pocas propiedades para el módulo Board. Algunas de las funciones claves de este módulo fueron testeadas a través de dichas propiedades. Por ejemplo, la siguiente:

\begin{lstlisting}[frame=single, language=haskell, captionpos=b, caption=Propiedad sobre los movimientos legales del rey]
-- Must be impossible for a king to move
to a position threatened by an enemy
prop_legal_king_movement =
  forAll validBoard $ \board ->
    forAll (arbitrary :: Gen Color) $ \color ->
      let kingPosition      =
            head $ positionsOf board color King
          theKingMovements  =
            freeMovements board color kingPosition
          possibleBoards    =
            map (\possibleDst ->
              placePiece board kingPosition possibleDst)
                theKingMovements
          safeMovement      =
            \possibleBoard -> not $
              isCheck possibleBoard color
       in all safeMovement possibleBoards
\end{lstlisting}

Establece que <<para todo>> tablero válido y <<para todo>> turno posible, los posibles tableros resultantes de ubicar al rey del bando que debe jugar en alguno de los casilleros donde dicho rey puede moverse tienen que ser todos seguros, es decir, no puede haber un movimiento posible para el rey en donde se deje a sí mismo en jaque.

Actualmente hay implementadas cinco propiedades -todas <<válidas>>-, con lo que el porcentaje de cobertura de los tests es bastante pobre. Los que están, sirvieron para encontrar varios problemas conceptuales en la implementación de algunas funciones, y de alguna manera dejan constancia de que la implementación actual no viola los invariantes planteados. Seguramente a futuro habrá que seguir pensando en diseñar más propiedades para probar.

\section{Conclusiones}

El desarrollo de este trabajo ha dejado un buen número de impresiones positivas y, algunas pocas, negativas. En principio, fue fructífero poder estudiar, analizar e implementar algunos de los conceptos de IA que son pilares en el desarrollo de un juego complejo como lo es el ajedrez. Muchos otros algoritmos y variantes quedaron fuera del estudio, pero podrían implementarse de forma independiente sin alterar ningún otro componente. En este sentido, la separación planteada entre la lógica propia del juego y el motor de IA a través del uso de clases de Haskell pareciera que dio buen resultado. Por otro lado, el estudio del lenguaje en sí y el desarrollo de esta aplicación fue un desafio más que interesante.

En cuanto a lo didáctico, el trabajo resultó en lineas generales placentero. En relación a esto, un problema que se ha notado es que en la bibliografia y en los textos en general dedicados a explicar y desarrollar las ideas de IA, lo hacen desde una perspectiva puramente operacional y resulta díficil entenderlo y <<traducirlo>> a un lenguaje más denotacional como el que sugiere Haskell. De hecho, una primera versión del motor de IA implementada para este trabajo se logró por medio de esta política y, posteriormente, en \cite{whyfunctional} y \cite{introfunctional} donde se muestran implementaciones de IA mucho más funcionales y elegantes, se decidió a última hora cambiar el diseño del motor hacia algo similar al estilo planteado en la bibliografía.

Sobre lo negativo, se puede decir que algunas de las reglas no quedaron bien definidas por haber sido demasiado simplista en la definición de tipos. Se intentó sobre la mitad del desarrollo intentar mejorar el diseño pero dada la falta de experiencia propia en el desarrollo sobre el lenguaje y la rigurosidad que el mismo impone -en especial con los tipos para que todas las expresiones queden coherentes-, se relegó dicha tarea y se continuó con el resto de las funcionalidades. Aun así, la experiencia resultó muy valiosa y seguramente servirá en una futura aplicación.

Como se ha mencionado en la introducción, la idea posterior es liberar el código fuente presente en este trabajo y, en lo posible, seguir desarrollandolo junto a la comunidad para seguir aprendiendo y aplicando conceptos sobre programación funcional.

%
% ---- Bibliography ----
%
\begin{thebibliography}{}
  \bibitem[1]{whyfunctional}
    J. Hughes, Why functional programming matters, The Computer Journal, v.32 n.2, p.98-107, April 1989 \href{https://dx.doi.org/10.1093/comjnl/32.2.98}{[doi$>$10.1093/comjnl/32.2.98]}
  \bibitem[2]{introfunctional}
    Richard Bird, Philip Wadler, An introduction to functional programming, Prentice Hall International (UK) Ltd., Hertfordshire, UK, 1988
  \bibitem[3]{ia_modernaproach}
    S.J. Russell, P. Norvig, Artificial Intelligence: A Modern Approach, Prentice Hall, Englewood Cliffs, NJ, 1995,.
  \bibitem[4]{parallel_and_concurrent}
    Simon Marlow, Parallel and Concurrent Programming in Haskell, O'Reilly Media, Sebastopol, California, USA, 2013
\end{thebibliography}
\end{document}
